\documentclass[12pt, a4paper, twoside]{scrartcl}

\usepackage[left=20mm,right=20mm,top=25mm,bottom=25mm,bindingoffset=10mm]{geometry}
\usepackage[T1]{fontenc}
% \usepackage[polish]{babel}
\usepackage[utf8]{inputenc}

\usepackage{mathptmx}
\usepackage{courier}
\usepackage[printonlyused]{acronym}
\usepackage{hyperref}
%\usepackage{url}
\usepackage{xurl}

\usepackage[normalem]{ulem}

\usepackage{sectsty}
\allsectionsfont{\normalfont\bfseries\fontsize{14}{18}\selectfont}

\usepackage[titles]{tocloft}
\setcounter{tocdepth}{2}
\renewcommand{\cftsecfont}{\fontsize{12}{14}\selectfont}
\renewcommand{\cftsecpagefont}{\fontsize{12}{14}\selectfont}
\renewcommand{\cftsecleader}{\fontsize{12}{14}\selectfont\cftdotfill{1}}
\renewcommand{\cftsecindent}{0mm}
\renewcommand{\cftsecnumwidth}{22mm}
\renewcommand{\cftsecaftersnum}{. }
\renewcommand{\cftbeforesecskip}{2mm}

\renewcommand{\cftsubsecfont}{\fontsize{11}{13}\selectfont}
\renewcommand{\cftsubsecpagefont}{\fontsize{11}{13}\selectfont}
\renewcommand{\cftsubsecindent}{5mm}
\renewcommand{\cftsubsecnumwidth}{8mm}
\renewcommand{\cftsubsecaftersnum}{. }
\renewcommand{\cftbeforesubsecskip}{2mm}
\renewcommand{\cftsubsecleader}{\fontsize{11}{13}\selectfont\cftdotfill{1}}

\renewcommand{\cftsubsubsecfont}{\fontsize{11}{13}\selectfont}
\renewcommand{\cftsubsubsecpagefont}{\fontsize{11}{13}\selectfont}
\renewcommand{\cftsubsubsecindent}{5mm}
\renewcommand{\cftsubsubsecnumwidth}{12mm}
\renewcommand{\cftsubsubsecaftersnum}{. }
\renewcommand{\cftbeforesubsubsecskip}{0mm}
\renewcommand{\cftsubsubsecleader}{\fontsize{11}{13}\selectfont\cftdotfill{1}}

\renewcommand{\cftfigindent}{0mm}
\renewcommand{\cftfignumwidth}{15mm}
\renewcommand{\cftfigpresnum}{Rys. }
\renewcommand{\cftfigaftersnum}{. }
\renewcommand{\cftfigleader}{\normalfont\cftdotfill{1}}

\renewcommand{\cfttabindent}{0mm}
\renewcommand{\cfttabnumwidth}{15mm}
\renewcommand{\cfttabpresnum}{Tab. }
\renewcommand{\cfttabaftersnum}{. }
\renewcommand{\cfttabfont}{\normalfont}
\renewcommand{\cfttableader}{\normalfont\cftdotfill{1}}

\usepackage{amsmath,amsfonts,amsthm}

\usepackage{color}
\usepackage{float}

\usepackage{verbatimbox}
\def\verbarg{{\fontsize{11}{13}\selectfont\makebox[7mm]{\arabic{VerbboxLineNo}}}\hspace{12mm}}

\usepackage[plain]{algorithm}
\floatstyle{plaintop}
\restylefloat{algorithm}
\floatname{algorithm}{Alg.}

\usepackage[pdftex]{graphicx}

\renewcommand\thesection{Chapter \Roman{section}}
\renewcommand\thesubsection{\Roman{section}.\arabic{subsection}}

\usepackage{longtable}
% \captionsetup[figure]{name=Rys., labelsep=period}
% \captionsetup[table]{name=Tab.,  labelsep=period}
% \captionsetup[lstlisting]{margin=0pt, font=bf, labelsep=period}
% \captionsetup[algorithm]{font=bf, labelsep=period}

\usepackage{listings}
\usepackage{color}

\usepackage{fancyhdr}
\pagestyle{fancy}
\fancyhead[RO, LE]{\thepage}
\fancyhead[RE, LO]{}
\fancyfoot{}
\renewcommand\headrulewidth{0pt}

\fancypagestyle{firststyle}{
\fancyhead[RO, LE]{}
\fancyhead[RE, LO]{}
}

\setlength\parindent{27pt}

\newcommand{\inserttitlepage}{
    \thispagestyle{firststyle}
    \begin{center}
        \fontsize{22}{14}\textbf{MILITARY~UNIVERSITY~OF~TECHNOLOGY}\\
        \fontsize{12}{10}\textbf{named after Jarosław Dąbrowski}
    \end{center}
    \vspace{-7mm}
    \rule{\linewidth}{0.5pt}
    \begin{center}
        \fontsize{20}{10}{\textbf{FACOULTY~OF~CYBERNETICS}}
    \end{center}

    \vspace{5mm}

    \begin{center}
        \includegraphics[width=40mm]{./WAT.png}
    \end{center}

    \vspace{0mm}

    \begin{center}
        {{\fontsize{26}{10} \selectfont {DIPLOMA THESIS}}}\\
        \vspace{3mm}
        {\fontsize{20}{10} \selectfont \stopien}
    \end{center}

    \vspace{5mm}

    \begin{longtable*}{p{.18\textwidth} p{.75\textwidth}}
        \fontsize{14}{20}\selectfont{Thesis topic:} &\fontsize{16}{20}\selectfont{\textbf{\noindent\temat}} \\
    \end{longtable*}

    \vspace{5mm}

    \begin{center}
        \textrm{\textbf{\fontsize{14}{20}\selectfont{\kierunek}}}\\
        \scriptsize\textrm{...........................................................................................................\\{{\scriptsize{(field of study)}}}}\\~\\
        \vspace{5mm}
        \textrm{\textbf{\fontsize{14}{20}\selectfont{\specjalnosc}}}\\
        \scriptsize\textrm{...........................................................................................................\\{{\scriptsize{(specialization)}}}}\\~\\
    \end{center}

    \vspace{5mm}

    \begin{longtable*}{p{.5\textwidth} p{.5\textwidth}}
        \textrm{Candidate:}\vspace{5mm} & 	\textrm{Supervisor:}\\
        \textrm{\textbf{\autor}}  & \textrm{\textbf{\promotor}} \\
    \end{longtable*}

    \begin{center}
    \rule{\linewidth}{0.5pt}
    \fontsize{12}{10} \textbf{\data} \end{center}


    \clearpage

    \thispagestyle{firststyle}

    \fontsize{14}{20} \selectfont{\textbf{OŚWIADCZENIE}}\\

    \begin{center}
        \fontsize{14}{20} \selectfont \textit{\ifdefined\zgoda  Wyrażam zgodę / \sout{nie wyrażam zgody} \else \sout{Wyrażam zgodę} / {nie wyrażam zgody} \fi $^{*}$\\na udostępnianie mojej pracy przez Archiwum WAT}
    \end{center}

    ~\\

    \begin{longtable*}{p{.6\textwidth} c}
    Dnia ...................... & 	....................................\\
    ~ & \fontsize{10}{10} \selectfont (podpis)\\
    \end{longtable*}
    \textit{$^{*}$ Niepotrzebne skreślić}

    % \begin{figure}[H]
    % \begin{center}
    %     \includegraphics[width=1\textwidth]{files/zgoda.jpeg}
    % \end{center}
    % \end{figure}

    \newpage
    \normalsize

    \tableofcontents
\usepackage[font=bf, labelfont=bf, belowskip=0pt]{caption}%

    \newpage
}

% \defcaptionname{polish}{\refname}{Bibliografia}

\definecolor{mygray}{rgb}{0.95,0.95,0.95}

% \renewcommand{\lstlistingname}{Kod.}

\lstset{
    basicstyle=\ttfamily\footnotesize,
    numbers=left,
    numberstyle=\ttfamily\footnotesize\color{black},
    numbersep=5pt,
    showspaces=false,
    breaklines=true,
    keepspaces=true,
    tabsize=2,
    xleftmargin=20pt,
    framexleftmargin=20pt,
    captionpos=tc,
    backgroundcolor=\color{mygray},
    keywordstyle=\color{black},
    inputencoding=utf8
}

\newcommand{\source}[1]{\vspace{0mm} \noindent\fontsize{10}{10}\selectfont{Źródło: {#1}}\normalsize\vspace{0mm}}

\usepackage{todonotes}
\newcommand{\TODO}[1]{\todo[color=red!50,inline]{\textbf{TODO}: #1}}
\newcommand{\kw}[1]{\todo[color=green!40,inline]{\textbf{KW}: #1}}
\newcommand{\ms}[1]{\todo[color=orange!40,inline]{\textbf{MS}: #1}}

\newcommand{\listequationsname}{List of Equations}
\newlistof{myequations}{equ}{\listequationsname}
\newcommand{\myequations}[1]{%
\addcontentsline{equ}{myequations}{\protect\numberline{\theequation}#1}\par}
\setlength{\cftmyequationsnumwidth}{2.5em}% Width of equation number in List of Equations

\usepackage{xcolor}

% Definicje zmiennych dla strony tytułowej
\newcommand{\kierunek}{COMPUTER SCIENCE}
\newcommand{\specjalnosc}{DATA ANALYSIS}
\newcommand{\stopien}{STUDIES II$^{\mathrm{o}}$} %Odpowiednie usunąć
\newcommand{\temat}{Wavelet neuronal net for chosen application on medical signals}
\newcommand{\data}{Warsaw 2025}
\newcommand{\autor}{eng. Radosław RELIDZYŃSKI}
\newcommand{\promotor}{prof. dr habil. eng. Andrzej WALCZAK}
\newcommand{\zgoda}{TAK} %W przypadku braku zgody zakomentować tą linię

\begin{document}


% Wywołanie strony tytułowej
\inserttitlepage


% Introduction
\section*{Introduction}
\addcontentsline{toc}{section}{Introduction}

The main task of my thesis is to perform some experiments to investigate how to combine wavelet transformation signal decomposition and convolutional neural networks to create a tool that will detect diseases in \ac{ECG} heartbeat signal.

\clearpage


% Theoretical introduction to the topic
\section{Theoretical introduction to the topic}

The following sub-chapters cover an introduction to the \ac{ECG} signal, the wavelet transform, and neural networks. In the subsequent chapters of this thesis, I will apply the acquired theoretical knowledge and integrate it to develop a wavelet neural network.

\subsection{\ac{ECG} signal}

Electrocardiography is the process of producing an electrocardiogram (ECG or EKG), a recording of the heart's electrical activity through repeated cardiac cycles.\cite{Wiki:electrocardiography}

\subsubsection{Electromagnetics impulses}

\subsubsection{\ac{QRS} complex}

\subsubsection{Heart diseases and their impact on \ac{ECG}}

\subsection{Wavelet transform}

\subsubsection{What is a wavelet?}

\subsubsection{What is a wavelet transform?}

\subsubsection{Wavelet transform on \ac{ECG} signal}

\subsubsection{What is \ac{DWT}?}

\subsection{What is \ac{CNN}?}

\subsection{What is \ac{WCNN}?}

\clearpage


% My plan for \ac{WCNN} creation
\section{My plan for \ac{WCNN} creation}

\subsection{Neural network architecture}

\subsection{Using \ac{DWT} in \ac{WCNN}}

\subsection{Learning process flow}

\clearpage


% Data source description
\section{Data source description}

\subsection{Continous \ac{ECG} signal}

\subsection{Source of data}

\clearpage


% My \ac{WCNN} realization
\section{My \ac{WCNN} realization}

\subsection{Techonogies used to create \ac{WCNN}}

\subsection{Creating \ac{WCNN}}

\subsection{Creating learning process}

\subsection{Creating full tool for the research}

\clearpage


% \ac{WCNN} implementation
\section{\ac{WCNN} implementation}

\subsection{Techonogies used to create \ac{WCNN}}

\subsection{Creating \ac{WCNN}}

\subsection{Creating learning process}

\subsection{Creating full tool for the research}

\clearpage


% Testing \ac{WCNN} model
\section{Testing \ac{WCNN} model}

\subsection{Learning process}

\subsection{Testing model}

confusion matrix

\subsection{Analysis of testing results}

\clearpage


\section*{Summary}

In short,...

\clearpage


% Bibliografia
\bibliography{references}
\bibliographystyle{plain}
\addcontentsline{toc}{section}{References}
\clearpage

\section*{Acronyms}
\addcontentsline{toc}{section}{Acronyms}
\begin{acronym}
    \acro{ECG}{Electrocardiography}
    \acro{WCNN}{Wavelet convolutional neural networks}
    \acro{QRS}{QRS complex (Q-, R- and S-waves; ventricular depolarization)}
    \acro{DWT}{Discrete wavelet transform}
    \acro{CNN}{Convolutional neural network}
\end{acronym}

\clearpage

\section*{Glossary}
\addcontentsline{toc}{section}{Glossary}
\clearpage

\listoffigures
\addcontentsline{toc}{section}{List of Figures}
\clearpage

\listoftables
\addcontentsline{toc}{section}{List of Tables}
\clearpage

% \renewcommand{\lstlistlistingname}{Spis kodów źródłowych}
\lstlistoflistings
\addcontentsline{toc}{section}{Listings}

\end{document}